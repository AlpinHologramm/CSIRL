\chapter*{Les 8 reines}

On dispose d'un échiquier de dimension $8 \times 8$, et de $8$ reines disposées
sur celui-ci. Sachant qu'aux échecs une reine peut se déplacer en ligne, en
colonne ou en diagonale, \textbf{comment disposer disposer les reines sur
  l'échiquier de sorte qu'aucune ne soit sur la trajectoire d'une autre ?} Et
comment trouver toutes les solutions possibles à ce problème ?

\setchessboard{
    showmover=false
}

\begin{figure}[h!]
  \centering
  \chessboard[
    addpieces=Qc3,
    pgfstyle=straightmove,
    linewidth=0.01em,
    color=red,
    markmoves={c3-c1,c3-c8,c3-a3,c3-h3},
    markmoves={c3-h8,c3-a1,c3-a5,c3-e1},
    addpieces=Qg5,
    pgfstyle=straightmove,
    linewidth=0.01em,
    color=blue,
    markmoves={g5-g1,g5-g8,g5-a5,g5-h5},
    markmoves={g5-d8,g5-h4,g5-c1,g5-h6},
  ]
\end{figure}

\encart{Matériel}{
  \begin{itemize}
  \item Un plateau quadrillé (on peut le dessiner sur une feuille) ;
  \item 8 pièces à poser sur le plateau pour représenter les reines.
  \end{itemize}
}

\newpage

\section*{Dimension du problème}

Pour se faire une idée de la dimension du problème, calculons le nombre de
configurations possibles pour 8 reines sur un échiquier :

\begin{equation}
    \frac{64!}{(64-8)! \times 8!} = 4426165368
\end{equation}

Nous sommes donc à la recherche de quelques solutions parmi plus de 4 milliards
de configurations possibles. Comme souvent, la solution stupide est de tester
toutes les configurations pour trouver les solutions. Dans un cas comme
celui-ci, un ordinateur sera assez puissant pour résoudre le problème en un
temps acceptable. En revanche, cette approche est exclue pour un humain. Nous
allons donc devoir procéder d'une manière plus intelligente ...

\section*{Résolution récursive}

Une manière de construire une solution est de procéder par étapes.

1 démarrer d'un échiquier vide
2 ajouter une reine sur une case libre
3 recommencer 2 tant que c'est possible
4 si on atteint 8 reines, on a trouvé une solution

% FIXME l'idée à faire passer : solution partielle, on ajoute un élément pour créer une nouvelle solution.

% Récursivité : nf (informatique) voir "récursivité"

