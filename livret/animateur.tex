\begin{frame}{Le coin de l'animateur : trucs et astuces pour s'assurer que le message passe bien}
\label{coin::animateur}

  Pour que le déroulement des activités se passent bien, voici quelques conseils.

  \begin{block}{Remarques générales}
    \begin{itemize}
    \item Appropriez vous les activités. Pratiquez les à l'avance et n'hésitez pas à ne pas suivre les consignes à la lettre.
    \item Ces activités sont des bases de discussion avec les participants, il n'y a pas d'évaluation à la fin.
    \item Evitez les introductions théoriques ; commencez par les activités, elles serviront de support pour discuter de la théorie.
    \end{itemize}
  \end{block}
  \begin{block}{À propos du jeu de Nim}

    L'objectif de cette activité est simplement d'introduire la notion d'algorithme comme stratégie gagnante pour un problème donné.
    \begin{itemize}
    \item Commencez par jouer avec les participants, sans dire qu'il y a un truc. Si vous jouez bien, vous gagnerez à tous les coups.
    \item Bien sûr, pour gagner, vous devez laisser votre adversaire commencer. S'il insiste pour ne pas commencer, vous pouvez toujours gagner en rattrapant la stratégie gagnante à la première erreur.
    \item Si un participant connaît déjà la stratégie gagnante du jeu, il pourra vous remplacer pour jouer avec les autres participants.
    \item Si vous n'êtes pas sûr d'appliquer correctement la stratégie gagnante, proposez un match en 3 (ou en 5 en cas de coup dur ;)
    \item Pour amener les participants à découvrir la stratégie gagnante, vous pouvez grouper les clous par 4, rendant ainsi l'astuce plus visible.
    \end{itemize}
  \end{block}
\end{frame}

\begin{frame}{Le coin de l'animateur : trucs et astuces pour s'assurer que le message passe bien}
  \begin{block}{À propos du jeu du crêpier psycho-rigide}
    L'objectif de cette activité est de trouver un algorithme et de le faire verbaliser par les participants.

    \begin{itemize}
    \item Expliquez les règles et demandez au participant de tenter de résoudre le problème ;
    \item si il bloque, conseillez-le. Par exemple :
      \begin{itemize}
        \item "essaye d'abord de mettre la grande crêpe en bas"
        \item "où doit se trouver la grande crêpe pour pouvoir l'amener en bas ?"
      \end{itemize}
    \item Quand le participant à trouvé l'algorithme, demandez lui de l'expliquer. 
    \end{itemize}
  \end{block}
%\url{http://interstices.info/jcms/n_52318/genese-dun-algorithme?hlText=cr\%C3\%A8pes}

  \begin{block}{À propos du base-ball multicolore}
    L'objectif de cette activité est d'introduire les notions de correction et performance des algorithmes.

    \begin{itemize}
    \item Il faut laisser les participants chercher un peu en les faisant verbaliser
    \item S'ils sont sur le point de trouver l'algo juste, on introduit très vite l'algo faux pour préserver un enchaînement logique: "oui, ok, mais je vais vous montrer une façon de faire rigolote"
    \item Quand l'algo juste est établi, et avant de parler de performance, on peut appliquer sur une variante :
      \begin{itemize}
      \item Chaque participant prend une couleur (une maison placée au sol entre ses pieds)
      \item Chaque participant (sauf 1) prend un bonhomme dans chaque main
      \item À chaque étape, celui qui a une main libre prend un bonhomme dans la main d'un voisin
      \item (attention, c'est fastidieux à 8 ou 9 couleurs, il vaut mieux faire deux rondes car l'algo semble $O(n^2)$)
      \end{itemize}
    \item Expérimentalement, l'algo qui tourne converge très souvent vers la solution à 5 maisons, mais converge souvent vers la boucle infinie quand il y a plus de couleurs. Ne tentez pas le diable ;)
    \item Dans la disposition linéaire, il est plus simple de mettre la couleur avec un seul bonhomme à une extrémité, et commencer par remplir la maison de l'autre extrémité. Sinon, on se retrouve avec une maison remplie de un seul au milieu, et il faut comprendre que la solution passe par le stockage temporaire d'un pion de la maison d'à coté sur le trou.
    \item Le discours sur le $O(n)$ est volontairement approximatif. On veut faire sentir les choses; faire un vrai cours prend une douzaine d'heures (cf. \url{http://www.loria.fr/~quinson/Teaching/TOP/}).
    \item Il serait intéressant de prouver effectivement la correction de l'algorithme linéaire, ainsi que de quantifier la probabilité de fonctionner de l'algo qui tourne en fonction du nombre de maisons
    \item Au passage, le crépier ne ressemble pas du tout aux tours de Hanoï: l'histoire ressemble un peu, mais la résolution est très différente (il y a $2^n-1$ étapes à Hanoï et $3\times n$ au crépier\ldots)
    \end{itemize}
  \end{block}
\end{frame}
