\begin{frame}{Le coin de l'animateur\\[-5pt]
\label{coin::animateur}
  {\large Trucs et astuces pour s'assurer que le message passe bien}}
  \begin{block}{Remarques générales}
    \begin{itemize}
    \item Il faut vous approprier les activités. N'hésitez pas à ne pas suivre
      les consignes à la lettre.\\
      Ces activités sont des bases de discussion avec les participants, il n'y
      a pas d'évaluation à la fin.
    \item Une question récurrente des participants est de savoir ce que
      l'animateur fait, en recherche.\\
      Pensez à préparer une présentation compréhensible de vos recherches, avec
      le domaine général, ses difficultés et applications et quelques mots de
      vos préoccupations propres.\\
      {\footnotesize Exemple: Le domaine de mon travail est le parallélisme:
        est ce que ranger sa chambre va plus vite à 2 ou 3? oui. à 3000? Non,
        on perd du temps à se coordonner. Et pourtant, la météo de demain est
        calculée en utilisant plusieurs milliers d'ordis en même temps, ce qui
        est difficile. Dans ce domaine, mon travail à moi est d'établir des
        instruments scientifiques (simulateurs ou parcs de machines),
        comparables aux télescopes ou microscopes des physiciens, et qui
        servent d'outils aux scientifiques du domaine.}
    \end{itemize}
  \end{block}
  \begin{block}{À propos du mot d'introduction}
    \begin{itemize}
    \item On peut faire cette présentation soit au début, soit juste après
      l'activité sur le jeu de Nim.
    \item Commencer directement par un petit jeu permet d'éviter que les
      participants ne décrochent avant même qu'on ne commence.
    \end{itemize}
  \end{block}
  \begin{block}{À propos du jeu de Nim}
    \begin{itemize}
    \item L'objectif de cette activité est simplement d'introduire la notion d'algorithme
    \item On propose le jeu avec le participant, mais sans dire trop vite qu'on
      a un truc. S'il y a plusieurs participants, on jouera avec plusieurs
      personnes, pour laisser sa chance à chacun. On peut faire une sorte de
      petit tournois.
    \item Il faut bien sûr laisser commencer le participant pour gagner à coup
      sûr. S'il insiste pour ne pas commencer, on peut le faire (et rattraper
      la stratégie gagnante à la première erreur du participant)
    \item On n'introduit l'existence du truc pour gagner que plus tard, quand
      on gagne à plate couture
    \item Si on perd, c'est à dire si on n'a pas réussi à appliquer la
      stratégie gagnante, il faut proposer un match en 3 (ou en 5 en cas de
      coup dur ;)
    \item On peut amener le participant à découvrir la stratégie gagnante en
      groupant les clous par paquets de 4 au lieu de la disposition pyramidale.
    \item Si l'un des participants connaît déjà la stratégie gagnante du jeu,
      il peut remplacer l'animateur dans une partie avec d'autres participants
    \end{itemize}
  \end{block}
\end{frame}

\begin{frame}{Le coin de l'animateur\\[-5pt]
  {\large Trucs et astuces pour s'assurer que le message passe bien}}
  \begin{block}{À propos du jeu du crêpier psycho-rigide}
    \begin{itemize}
    \item L'objectif de cette activité est de trouver un algorithme et de
    le faire verbaliser par les participants
    \item On propose au participant de d'abord tenter de le résoudre
    intuitivement, sans réfléchir
    \item Si le participant bloque, on peut lui donner un conseil : \og~Une
    bonne première étape est de se débrouiller pour mettre la grande en bas~\fg
    \item Si le participant bloque toujours, on peut lui donner un second
    conseil : \og~où est-ce que la grande devrait être pour pouvoir la mettre en
    bas ? ~\fg puis le guider pour l'étape suivante.
    \item On essaie ensuite de faire expliquer l'algorithme par le participant.
    On gagne à ce que ce soit le participant et non l'animateur qui explique aux
    autres, avec ses propres mots.
    \end{itemize}
  \end{block}
%\url{http://interstices.info/jcms/n_52318/genese-dun-algorithme?hlText=cr\%C3\%A8pes}
\end{frame}

\begin{frame}{Le coin de l'animateur\\[-5pt]
  {\large Trucs et astuces pour s'assurer que le message passe bien}}
  \begin{block}{À propos du base-ball multicolore}
    \begin{itemize}
    \item L'objectif de cette activité est d'introduire les notions de
      correction et performances d'algorithmes
    \item Il faut laisser les participants chercher un peu en les faisant verbaliser
    \item S'ils sont sur le point de trouver l'algo juste, on introduit très
      vite l'algo faux pour préserver un enchaînement logique: "oui, ok, mais
      je vais vous montrer une façon de faire rigolote"
    \item Quand l'algo juste est établi, et avant de parler de performance, on
      peut appliquer sur une variante:\vspace{-\baselineskip}
      \begin{itemize}
      \item Chaque participant prend une couleur (une maison placée au sol entre ses pieds)
      \item Chaque participant (sauf 1) prend un bonhomme dans chaque main
      \item À chaque étape, celui qui a une main libre prend un bonhomme dans la main d'un voisin
      \item (attention, c'est fastidieux à 8 ou 9 couleurs, il vaut mieux faire
        deux rondes car l'algo semble $O(n^2)$)
      \end{itemize}
    \item Expérimentalement, l'algo qui tourne converge très souvent vers la
      solution à 5 maisons, mais converge souvent vers la boucle infinie quand
      il y a plus de couleurs. Ne tentez pas le diable ;)
    \item Dans la disposition linéaire, il est plus simple de mettre la couleur
      avec un seul bonhomme à une extrémité, et commencer par remplir la maison
      de l'autre extrémité. Sinon, on se retrouve avec une maison remplie de un
      seul au milieu, et il faut comprendre que la solution passe par le
      stockage temporaire d'un pion de la maison d'à coté sur le trou.
    \item Le discours sur le $O(n)$ est volontairement approximatif. On veut
      faire sentir les choses; faire un vrai cours prend une douzaine d'heures
      (cf. \url{http://www.loria.fr/~quinson/Teaching/TOP/}).
    \item Il serait intéressant de prouver effectivement la correction de
      l'algorithme linéaire, ainsi que de quantifier la probabilité de
      fonctionner de l'algo qui tourne en fonction du nombre de maisons
    \item Au passage, le crépier ne ressemble pas du tout aux tours de Hanoï:
      l'histoire ressemble un peu, mais la résolution est très différente (il y
      a $2^n-1$ étapes à Hanoï et $3\times n$ au crépier\ldots)
    \end{itemize}
  \end{block}
\end{frame}

%%% Local Variables: 
%%% mode: latex
%%% TeX-master: "CSIRL"
%%% End: 
