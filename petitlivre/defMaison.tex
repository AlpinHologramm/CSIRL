\newcommand{\maisonPair}[5]{ \begin{tikzpicture}
  \node[name=m,shape=regular polygon,regular polygon sides=#3,minimum size=18mm,rotate=(360/#3)]{};
  \node[name=b,shape=regular polygon,regular polygon sides=#4,minimum size=12mm,rotate=(360/#4)/2]{};
  \foreach \base/\maison in {#5} {
    \draw[shift=(m.corner \base)]
       node[shape=ellipse,fill=\maison,draw=black,rotate=((360/#3)*(\base-1))+(360/#3/2)] {~~~~~};
  }
  \foreach \bb in {1,...,#4} {
    \draw[shift=(b.corner \bb)] node[name=bb \bb]{};
  }
  \foreach \base/\maison in {#1} {
    \draw[shift=(b.corner \base)]
       node[name=bb \base,shape=circle,fill=\maison,draw=black,inner sep=.1]
       {~~~};
%         {\footnotesize\base};
  }
  #2
\end{tikzpicture} }
\newcommand{\maisonImpair}[5]{ \begin{tikzpicture}
  \node[name=m,shape=regular polygon,regular polygon sides=#3,minimum size=18mm, inner sep=0pt]{};
  \node[name=b,shape=regular polygon,regular polygon sides=#4,minimum size=12mm]{};
  \foreach \base/\maison in {#5} {
    \draw[shift=(m.corner \base)]
       node[shape=ellipse,fill=\maison,draw=black,rotate=(360/#3)*(\base-1),name=m \base] {~~~~~};
  }
  \foreach \base/\maison in {#1} {
    \draw[shift=(b.corner \base)]
       node[shape=circle,fill=\maison,draw=black,inner sep=.1] {~~~};
  }
  \foreach \bb in {1,...,#4} {\draw[shift=(b.corner \bb)] node[name=bb \bb] {};}



%  \foreach \bb in {1,...,#4} {\draw[shift=(b.corner \bb)] node[name=bb \bb]{\tiny\bb};}%debug the bonshommes names
  #2
\end{tikzpicture} }

\newcommand{\maisonQuatre}[2]{\maisonPair{#1}{#2}{4}{12}{1/A,2/B,3/C,4/D}}
\newcommand{\maisonCinq}[2]{\maisonImpair{#1}{#2}{5}{20}{1/A,2/B,3/C,4/D,5/E}}
\newcommand{\maisonSix}[2]{\maisonPair{#1}{#2}{6}{24}{1/A,2/B,3/C,4/D,5/E,6/F}}
\newcommand{\maisonSept}[2]{\maisonImpair{#1}{#2}{7}{28}{1/A,2/B,3/C,4/D,5/E,6/F,7/G}}

\colorlet{A}{green!60}
\colorlet{B}{red!80}
\colorlet{C}{purple!40}
\colorlet{D}{black!2!yellow}
\colorlet{E}{blue!70}
\colorlet{F}{orange!80}
\colorlet{G}{olive}
\colorlet{H}{magenta}
\colorlet{I}{lime}
\colorlet{J}{pink}

\newcommand{\pawn}[1]{\tikz \draw node[shape=circle,fill=#1,draw=black,inner sep=.1] {~~~};}

